\chapter{Applicazione mobile}

L’interfaccia della piattaforma eSERVANT verso gli utenti finali è come da progetto una APP per mobile.
I due screenshot nella Figura 4 rappresentano, in ordine, la pagina di atterragio per gli utenti che
non hanno ancora effettuato l'accesso e la home dove si radudano tutti gli eventi seguiti dall'utente.

Per questo insieme a Sintra Consulting s.r.l. siamo andati a progettare in modo molto avanzato tutte le interfacce andando a quadrare requisiti di sistema/casi di uso con mockup montati all’interno di una vera e proprio demo dell’APP finale attesa.
Tale strumento ci ha permesso di valutare non solo la user experience degli utenti, ma anche di poter fornire agli sviluppatori uno strumento specifico sul quale andare a traguardare la loro attività.

\begin{figure}[htp]
    \centering  
    \caption{Screenshot applicazione mobile eServant}
    \subfloat[Login]{\includegraphics[scale=0.15]{img/cap2/1}}
    \subfloat[Home]{\includegraphics[scale=0.15]{img/cap2/2}}
\end{figure}

Il prototipo di APP è quindi stato creato andando a realizzare dei mockup sui sei temi di fondo:
\begin{itemize}
\item presentazione e riconoscimento;
\item gestione evento e socialità;
\item gestione delle funzioni di aiuto;
\item gestione delle funzioni di mobilità;
\item gestione delle funzioni di navigabilità e mappe;
\item gestione del proprio profilo.
\end{itemize}
\paragraph{}
\paragraph{}
\paragraph{}
\section{Framework Ionic}

Ionic è un HTML5 SDK lanciato nella sua prima versione nel 2013 e progettato per essere la base per lo sviluppo di app mobili ibride. 
Sviluppare un'app mobile ibrida significa scrivere un unico code base per i due OS più diffusi: IOS e Android.

Ionic viene distribuito come pacchetto installabile con npm, il package manager di Nodejs.
Node.js è un ambiente di run-time JavaScript open-source e multipiattaforma che esegue codice JavaScript all'esterno di un browser, lato server per esempio.

\begin{figure}[h!]
    \centering  
    \caption{Ionic framework [11]}
    \includegraphics[scale=0.4]{img/cap2/angular-ionic}
\end{figure}

Ionic si installa tramite CLI Linux come di seguito [12]
\begin{lstlisting}
npm install -g ionic
\end{lstlisting}

La struttura di Ionic è costituita da Angular, Apache Cordova (ex-PhoneGap) e SASS. Ciò consente la creazione di applicazioni mobili ricche di funzionalità che utilizzano esclusivamente tecnologie web come mostrato in Figura 5.
\\\\
\textbf{Cordova}
(\textit{Apache Cordova, ex-PhoneGap e futuro Capacitor})

\begin{figure}[h!]
    \centering  
    \caption{Cordova [13]}
    \includegraphics[scale=0.80]{img/cap2/cordova}
\end{figure}

Apache Cordova è un framework di sviluppo di applicazioni mobili che può essere utilizzato per creare app mobili multipiattaforma usando HTML5 e puro JavaScript. 
Per multipiattaforma intendiamo che il codice di un'applicazione può essere scritto una volta sola utilizzando HTML5 e JavaScript e può essere eseguito su diversi OS, come Android, iOS o Windows Mobile.
La Figura 6 raffigurara l'astrazione dell'implementazione nativa tramite l'uso di HTML, CSS e JS.
Riporto di seguito un frammento di codice che permette di aprire la fotocamera nativa del dispositivo mobile per scattare foto [14].
\\
\begin{lstlisting}
ionic cordova plugin add cordova-plugin-camera
npm install @ionic-native/camera
\end{lstlisting}

\begin{lstlisting}
import { Camera, CameraOptions } from '@ionic-native/camera/ngx';

const options: CameraOptions = {
  quality: 100,
  destinationType: this.camera.DestinationType.FILE_URI,
  encodingType: this.camera.EncodingType.JPEG,
  mediaType: this.camera.MediaType.PICTURE
}

constructor(private camera: Camera) { }

openCamera(){
    this.camera.getPicture(options).then((imageData) => {
        // imageData is either a base64 encoded string or a file URI
        // If it's base64 (DATA_URL):
        let base64Image = 'data:image/jpeg;base64,' + imageData;
        }, (err) => {
        // Handle error
    });
}
\end{lstlisting}

Vediamo come in sole 16 linee di codice siamo riusciti ad aprire la camera nativa del sistema operativo per scattare
una foto, ovviamente lo stesso codice viene compilato sia su IOS che su Android grazie a Cordova.
\\\\
\textit{SASS}\\
Scrivere CSS (Cascading Style Sheets) è fondamentale per descrivere in modo efficace il modo in cui gli elementi HTML devono 
essere visualizzati su una pagina Web per definire stili, design, layout e tutto ciò che è necessario per creare un sito 
Web sbalorditivo. Ma quando inizi a lavorare con siti grandi e complessi, potresti iniziare a chiederti se i CSS potrebbero 
essere migliori. 

In questi casi entra in gioco SASS (Syntacticly Awesome Stylesheets).
SASS [15] è un pre-processore CSS che consente di utilizzare variabili, operazioni matematiche, mixin, loop, funzioni, importazioni 
e altre funzionalità interessanti che rendono la scrittura CSS molto più potente. In un certo senso, si potrebbe pensare a 
SASS come a un linguaggio di estensione del foglio di stile perché estende le caratteristiche CSS standard introducendo i
vantaggi di un linguaggio di programmazione di base. Quindi SASS compilerà il codice e genererà l'output CSS che un
browser può comprendere. \\
Vediamo alcune funzioni SASS in confronto con CSS puro:\\\\

\textit{Variabili}
\begin{multicols}{2}
    \begin{lstlisting}
// SASS
$font-stack:    Helvetica, sans-serif
$primary-color: #333

body
    font: 100% $font-stack
    color: $primary-color
    \end{lstlisting}
    \columnbreak
    \begin{lstlisting}
// CSS
body {
    font: 100% Helvetica, sans-serif;
    color: #333;
} 

.
    \end{lstlisting}
\end{multicols}

\begin{multicols}{2}
    \columnbreak
\end{multicols}
\textit{Mixin}
\begin{multicols}{2}
    \begin{lstlisting}
// SASS
@mixin transform($property) 
    -webkit-transform: $property
    -ms-transform: $property
    transform: $property

.box
    @include transform(rotate(30deg))   
    \end{lstlisting}
    \columnbreak
    \begin{lstlisting}
// CSS
.box {
    -webkit-transform: rotate(30deg);
    -ms-transform: rotate(30deg);
    transform: rotate(30deg);
}

.
    \end{lstlisting}
\end{multicols}


\paragraph{}
\paragraph{}
\paragraph{}

\textit{Loop}


\begin{multicols}{2}
    \begin{lstlisting}
// SASS
$class-slug: for !default

@for $i from 1 through 4
    .#{$class-slug}-#{$i}
    width: 60px + $i









    .
    \end{lstlisting}

    \columnbreak
    \begin{lstlisting}
// CSS
.for-1 {
    width: 61px;
}

.for-2 {
    width: 62px;
}

.for-3 {
    width: 63px;
}

.for-4 {
    width: 64px;
}
    \end{lstlisting}   
\end{multicols}
   

\subsection{NgRx gestore stato}
\paragraph{}
NGRX è un gruppo di librerie per Angular ispirate al pattern Flux sviluppato dal team di Facebook per gestire
lo stato delle applicazioni client.

Lo scopo principale di questo pattern è fornire uno stato dell'applicazione prevedibile, basato su tre principi principali [16].

\begin{itemize}
    \item \textbf{Unica sorgente di verità}: significa che l'intero stato dell'applicazione è memorizzato un un'unica struttura dati come mostrato in Figura 9;
    \item \textbf{Lo stato è read only}: lo stato non viene mai cambiato direttamente, 
    solo tramite \textit{action}. Le \textit{action} descrivono cosa sta succedendo (possiamo vedere le action
    come richieste di ottenimento dati, rimozione, aggiornamento stato). Possiamo notare questa immediata differenza confrontando la Figura 7 con la Figura 8;
    \item \textbf{I cambiamenti sono fatti solo da funzioni pure}: quando un'\textit{action} viene lanciata, essa verrà
    intercettata dai \textit{reducer}.
    Questi \textit{reducer} (sono solo funzioni pure) ricevono un'azione e lo stato, quindi a seconda dell'azione inviata 
    (di solito con un'istruzione switch), eseguono un'operazione e restituiscono un nuovo oggetto di stato. 
    Lo stato in un'app redux è immutabile! Quindi, quando un \textit{reducer} modifica qualcosa nello stato, 
    restituisce un nuovo oggetto stato;
\end{itemize}

\paragraph{}
Possiamo renderci conto dei tre principi fondamentali sopra citati confrontando la Figura 7 con la Figura 8.

\begin{figure}[h!]
    \centering  
    \caption{App con gestore stato NgRx [17]}
    \includegraphics[scale=0.4]{img/cap2/ngrx}
\end{figure}
\paragraph{}

\begin{figure}[h!]
    \centering  
    \caption{Senza gestore stato NgRx}
    \includegraphics[scale=0.5]{img/cap2/without-ngrx}
\end{figure}
\paragraph{}

\begin{figure}[h!]
    \centering  
    \caption{Ispezione stato NgRx}
    \includegraphics[scale=0.29]{img/cap2/ngrx-eservant-90}
\end{figure}

\section{Social login}
Al primo accesso all'APP eServant è necessario dare il consenso per almeno uno dei seguenti social:
\begin{itemize}
\item Facebook;
\item Instagram;
\item Twitter.
\end{itemize}

Altrimenti viene data la possibilità di fare un accesso privato tramite username e password.
L'autenticazione verso i social network sopra citati avviene tramite protocollo OAuth2.0.

Il framework di autorizzazione OAuth2.0 abilita un'applicazione di terze parti a ottenere un accesso limitato a un servizio HTTP, attivo
per conto di un proprietario di risorse, orchestrando un'interazione di approvazione tra il proprietario della risorsa e il servizio HTTP.
Questa specifica sostituisce il protocollo OAuth 1.0 descritto in RFC 5849.

\begin{figure}[h!]
    \centering  
    \caption{OAuth 2.0 flow [18]}
    \includegraphics[scale=0.60]{img/cap2/oauth20-2}
\end{figure}

\paragraph{}

Vediamo quali sono gli \textbf{Attori} coinvolti nel protocollo OAuth2.0 [18] mostrato in Figura 10.

\begin{itemize}
\item \textit{resource owner}: un'entità in grado di approvare l'accesso a una risorsa protetta;
\item \textit{authorization server}: il server che invia i token di accesso al client dopo essere stato autorizzato
dal resource owner;
\item \textit{resource server}: il server che ospita le risorse protette. Il resource Server è in grado di accettare
e rispondere alle richieste di risorse protette utilizzando i token di accesso;
\item \textit{client}: un'applicazione che fa richieste di risorse protette per conto del
proprietario delle risorse e con la sua autorizzazione. Il termine "client"
non implica particolari caratteristiche di implementazione (l'applicazione può essere eseguita
su un server, un client web o altri dispositivi).
\end{itemize}


\section{Navigazione impianto}
Grazie alla collaborazione con il MICC di Firenze siamo riusciti ad integrare una cartografia basata su OpenStreetMap
per permettere ai partecipanti di eventi la navigazione verso un determinato POI (point of interest) partendo
dall'ultima posizione rilevata del partecipante.

Il nostro compito si è suddiviso in tre parti:
\begin{itemize}
\item Creazione livelli da applicare alle mappe OSM per la navigazione interna;
\item Integrazione delle mappe sia sull'APP Ionic che sul backoffice web;
\item Recupero ultima posizione dell'utente;
\end{itemize}

L'integrazione all'interno dell'app Ionic è stata fatta
tramite iframe.
Frammento di codice dell'app Ionic per includere le mappe:
\begin{lstlisting}[language=html]
<iframe src="https://www.eservant.it/maps"></iframe>
\end{lstlisting}

\paragraph{}
Il frammento indicato sopra permette tramite una singola linea di codice
di includere le mappe e ottenre il risultato come da screenshot in Figura 11.
\begin{figure}[htp]
    \centering  
    \caption{Mappe nell'app eServant}
    \subfloat[Selezione POI]{\includegraphics[scale=0.5]{img/cap2/maps-1}}
    \subfloat[Avvio navigazione verso il POI]{\includegraphics[scale=0.5]{img/cap2/maps-2}}
\end{figure}


\section{Geolocalizzazione}
Come abbiamo detto nella sezione precedente, la navigazione all'interno dell'impianto necessita la posizione iniziale
dalla quale far iniziare il percorso verso il POI selezionato.
Questa posizione ovviamente è un elemento dinamico che deve mutare nel tempo dipendentemente dalla posizione
dell'utente all'interno dell'impianto.

Da un primissimo brainstorming è emerso che la tecnologia GPS non è affidabile in impianti indoor ma anche in ambienti
outdoor potrebbe fallire; per questo principale motivo è stata creata una gerarchia di tecnologie da usare per
la Geolocalizzazione dell'utente, dalla più affidabile alla meno affidabile.

In ordine di priorità di affidabilità decrescente, troviamo:

\begin{itemize}
    \item QRcode manuale;
    \item GPS;
    \item iBeacon.
\end{itemize}


È possibile vedere il risultato della geolocalizzazione nelle mappe
nella Figura 12.
\begin{figure}[H]
    \caption{Localizzazione nelle mappe}
    \centering  
    \includegraphics[scale=0.5]{img/cap2/geo-1}
\end{figure}


\subsection{QRcode}
\begin{figure}[H]
    \centering  
    \caption{QRcode}
    \includegraphics[scale=1]{img/cap2/barcode-qrcode}
\end{figure}
I codici QR (esempio in Figura 13) sono stati creati nel 1994 da Denso Wave [19], una filiale giapponese del Gruppo Toyota.
L'uso di questa tecnologia è ora gratuito. Il QR Code non è l'unico codice a barre bidimensionale 
sul mercato, un altro esempio è il codice Data Matrix.
\\
Un codice QR è un codice a barre quadrato bidimensionale che può memorizzare dati codificati. 
Il più delle volte i dati sono un link a un sito web (URL).

Nel caso di eServant il codice QRcode racchiude un identificativo di un POI. \\ \\
Vediamo le fasi del processo di geolocalizzazione tramite QRcode:

\begin{enumerate}
    \item Scansione QRcode tramite l'app di eServant.\\
    Sempre tramite cordova, un esempio di accesso alla camera nativa dell'OS per scansionare
    codici QRcode [20];
\begin{lstlisting}
    import { BarcodeScanner } from '@ionic-native/barcode-scanner/ngx';
    
    constructor(private barcodeScanner: BarcodeScanner) { }
    
    this.barcodeScanner.scan().then(barcodeData => {
        console.log('Barcode data', barcodeData);
    }).catch(err => {
        console.log('Error', err);
    });
\end{lstlisting}

    \item Decodifica del QRcode e ottenimento del POI ID;
    \item Invio del POI ID, insieme all' ID dell'utente, al server;
    \item Il server, una volta ottenute le coordinate del POI, provvede ad aggiornare l'ultima posizione dell'utente;
\end{enumerate}


\subsection{A-GPS}

Tramite GPS assistito (A-GPS) la localizzazione del partecipante all'evento è automatica,
a patto che il GPS del dispositivo mobile sia abilitato. 

La tecnologia GPS assistito permette di acquisire segnale satellitare in maniera più rapida
rispetto al GPS tradizionale.

Tramite rete dati cellulare vengono indicate al dispositivo mobile le posizioni previste dei
satelliti GPS come mostrato in Figura 14. In questo modo, il dispositivo sa dove cercare i satelliti ed è in grado di acquisirne 
i segnali in pochi secondi, anche in condizioni difficili.
\begin{figure}[H]
    \centering  
    \caption{Comunicazione A-GPS [21]}
    \includegraphics[scale=0.95]{img/cap2/gps}
\end{figure}
Vediamo un frammento di codice per ottenere in tempo reale la 
posizione del dispositivo tramite A-GPS [22].
\begin{lstlisting}
import { Geolocation } from '@ionic-native/geolocation/ngx';

constructor(private geolocation: Geolocation) {}

this.geolocation.getCurrentPosition().then((resp) => {
    // resp.coords.latitude
    // resp.coords.longitude
}).catch((error) => {
    console.log('Error getting location', error);
});

let watch = this.geolocation.watchPosition();
watch.subscribe((data) => {
    // data can be a set of coordinates, or an error (if an error occurred).
    // data.coords.latitude
    // data.coords.longitude
});
\end{lstlisting}

\subsection{iBeacon}

\begin{figure}[H]
    \centering  
    \caption{Dispositivo iBeacon [24]}
    \includegraphics[scale=0.3]{img/cap2/beacon}
\end{figure}

iBeacon [23] è una tecnologia basata su Bluetooth Low Energy. A partire dal 2013 iBeacon è 
integrato in Apple iOS 7. Il primo progetto pilota è stato lanciato nei negozi Apple nel
Dicembre 2013.

\textbf{BLE (Bluetooth Low Energy)}
Bluetooth Low Energy (BLE) è uno standard di trasmissione radio sviluppato dalla community 
Bluetooth SIG. Le caratteristiche di questo standard sono mirate a soddisfare le esigenze
delle moderne applicazioni wireless, come un consumo energetico estremamente basso, una
connessione corta che scandisce i tempi, affidabilità e sicurezza. 
BLE consuma da 10 a 20 volte meno energia (e 1000 volte meno del Wi-Fi), e può trasmettere
dati con velocità 50 volte superiore rispetto al Bluetooth classico. 

Un esempio di dispositivo iBeacon è mostrato in Figura 15.\\
Lato client (Ionic), il plugin cordova IBeacon mette a disposizione due principali metodi
che permettono di intercettare due eventi:

\begin{itemize}
    \item Dispositivo mobile entrato nel raggio dell'ibeacon;
    \item Dispositivo mobile uscito dal raggio dell'ibeacon;
\end{itemize}

La Figura 16 mostra un dispositivo mobile entrato nel raggio di quattro iBeacon.
\begin{figure}[H]
    \centering  
    \caption{Prossimità tramite iBeacon [25]}
    \includegraphics[scale=0.3]{img/cap2/beacon-proximity}
\end{figure}


Frammento di codice Ionic [26]:
\begin{lstlisting}
import { IBeacon } from '@ionic-native/ibeacon/ngx';

constructor(private ibeacon: IBeacon) { }

// create a new delegate and register it with the native layer
let delegate = this.ibeacon.Delegate();

// Subscribe to some of the delegate's event handlers
delegate.didRangeBeaconsInRegion()
  .subscribe(
    data => console.log('didRangeBeaconsInRegion: ', data),
    error => console.error()
  );
delegate.didStartMonitoringForRegion()
  .subscribe(
    data => console.log('didStartMonitoringForRegion: ', data),
    error => console.error()
  );
delegate.didEnterRegion()
  .subscribe(
    data => {
      console.log('didEnterRegion: ', data);
    }
  );

let beaconRegion = this.ibeacon.BeaconRegion('deskBeacon','F7826DA6-ASDF-ASDF-8024-BC5B71E0893E');

this.ibeacon.startMonitoringForRegion(beaconRegion)
  .then(
    () => console.log('Native layer received the request to monitoring'),
    error => console.error('Native layer failed to begin monitoring: ', error)
  );
\end{lstlisting}

\section{Chatbot}

Il chatbot agisce come supporto automatico che, tramite una chat integrata nell'applicazione, cerca di rispondere
alle domande dell'utilizzatore.
Il motore dietro alla chat integrata è un NLP engine (Natural Language Processor).
Il suo compito è, per ogni stringa di testo proveniente dalla chat integrata sull'app, cercare di capire
il significato semantico della frase.

Introduciamo 2 concetti principali su cui si basa il motore NLP:

\begin{itemize}
\item Intent: rappresenta l'azione presente nella frase elaborata, è generalmente un verbo.
\item Slot: rappresenta uno o più attributi che si legano all'intent. In molti casi l'intent è lo stesso,
cambiano solo i valori degli slot.
\end{itemize}

Vediamo nel caso specifico di eServant rappresentato dagli screenshot in Figura 17.

\textit{Vorrei \textcolor{green}{raggiungere il punto di interesse} \textcolor{purple}{Aula Magna}}\\
Analisi tramite NLP:
\begin{itemize}
\item \textbf{intent}: \textcolor{green}{raggiungere il punto di interesse}
\item \textbf{slot}: 
\begin{itemize}
    \item Tipo slot: POI
    \item Valore: \textcolor{purple}{Aula Magna}
\end{itemize}
\end{itemize}
    

\begin{figure}[htp]
    \centering  
    \caption{Chatbot in esecuzione}
    \subfloat[Domanda]{\includegraphics[scale=0.5]{img/cap2/chatbot-1}}
    \subfloat[Risposta]{\includegraphics[scale=0.5]{img/cap2/chatbot-2}}
\end{figure}

\paragraph{}
\paragraph{}
\paragraph{}
\paragraph{}
\paragraph{}

\subsection{Brainstorming}
\begin{figure}[htp]
    \centering  
    \caption{Brainstorming per l'implementazione del chatbot [27][28][29][30]}
    \subfloat{\includegraphics[scale=0.5]{img/cap2/eservant-chatbot}}
\end{figure}


Come si nota dal flow decisionale in Figura 18, la prima domanda che ci siamo posti è stata se volevamo fare il deploy del
chatbot su cloud oppure in-house.\\

\subsection{Implementazione}

\begin{figure}[htp]
    \centering  
    \subfloat[Flow di decisione]{\includegraphics[scale=0.5]{img/cap2/eservant-chatbot-green}}
\end{figure}

Visto che il progetto eServant racchiude informazioni private degli impianti, abbiamo deciso di usare un motore
NLP in-house e non cloud.
Date le conoscenze in Quid Informatica del linguaggio di programmazione Python abbiamo deciso di procedere
con l'implementazione del chatbot usando il framework Rasa, scritto appunto in Python.

Rasa è un framework open source in Python per la creazione di chatbot basato su machine learning.
Invece che basarsi su una cascata di if/else, Rasa risponde usando un modello, creato attraverso
librerie di machine learning in Python sulla base di conversazioni campione.
