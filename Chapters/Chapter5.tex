\chapter{Conclusioni}

Prima di iniziare il mio percorso in Quid Informatica SPA, dove ho contribuito allo
sviluppo del progetto eServant, ho lavorato due anni nella Web Agency fiorentina Aperion SRL.
Nel periodo della web agency mi sono dedicato per passione ai servizi cloud di Amazon
Web Services che mi hanno portato alla finale della competizione mondiale "City on a
cloud", sempre di Amazon Web Services, con il progetto personale \textit{PDENSITY}.

Unendo questa mia esperienza con le alte competenze di Quid Informatica SPA sono 
riuscito ad immergermi a 360° nel progetto di ricerca eServant.\\

La sicurezza sociale oggi giorno è un tema molto acceso su cui dobbiamo
impegnarci a creare strumenti efficaci per salvaguardarla.

L'uso dei braccialetti bluetooth low energy in eServant sono un esempio di questi.
Le allerte automatiche quando la densità delle persone in un determinato varco
supera la soglia limite ne sono un altro.

Sicuramente c'è ancora molta strada da fare per mettere in sicurezza la vita 
sociale, in questo caso legata alla partecipazione di eventi pubblici. eServant 
è una prima soluzione concreta realizzata dal capofila di progetto Quid Informatica SPA.

Ricollegandomi un pò a quella che è stata la mia antecedente esperienza con "City on
a cloud" di Amazon Web Services, oggi giorno abbiamo la grande fortuna di avere
una serie di strumenti ad alto valore tecnologico, aggiungerei PlugAndPlay, pronti
all'uso.
Questo per dire che non dobbiamo più preoccuparci di inventare a basso livello 
nuove intelligenze, come per riconoscimento facciale o altro più avanzato, ma 
concentrarci sul migliorare il processo assemblando e configurando tecnologie
ormai consolidate.
